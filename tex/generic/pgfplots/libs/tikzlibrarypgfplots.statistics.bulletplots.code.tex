%--------------------------------------------
%
% Package pgfplots, library for statistical plots (boxplots in the first version)
%
% Copyright 2007-2014 by Christian Feuersänger.
%
% This program is free software: you can redistribute it and/or modify
% it under the terms of the GNU General Public License as published by
% the Free Software Foundation, either version 3 of the License, or
% (at your option) any later version.
% 
% This program is distributed in the hope that it will be useful,
% but WITHOUT ANY WARRANTY; without even the implied warranty of
% MERCHANTABILITY or FITNESS FOR A PARTICULAR PURPOSE.  See the
% GNU General Public License for more details.
% 
% You should have received a copy of the GNU General Public License
% along with this program.  If not, see <http://www.gnu.org/licenses/>.
%
%--------------------------------------------

%%%%%%%%%%%%%%%%%%%%%%%%%%%%%%%%%%%%%%%%%%%%%%%%%%%%%%%5
%
% Bullet graphs
% http://www.perceptualedge.com/articles/misc/Bullet_Graph_Design_Spec.pdf
%
%%%%%%%%%%%%%%%%%%%%%%%%%%%%%%%%%%%%%%%%%%%%%%%%%%%%%%%5


\def\pgfplots@bullet@dir{%
	\pgfkeysvalueof{/pgfplots/bullet graph/draw direction}%
}
\def\pgfplots@bullet@other@dir{%
	\if\pgfplots@bullet@dir x%
		y%
	\else
		x%
	\fi
}%

\def\pgfplots@bullet@min@invisible{-0.5}
\def\pgfplots@bullet@max@invisible{0.5}

\pgfqkeys{/pgfplots/bullet graph}{
	% The featured measure:
	performance/.initial=,
	%
	% Accepts a comma-separated list of interval midpoints like
	% range midpoints={200,250}. '<min>' and '<max>' are inserted
	% automatically.
	range midpoints/.initial=,
	%
	% any reference value(s):
	reference/.initial=,
	%
	% Defines the direction of the bullet plot.
	% Values are 'x' or 'y':
	draw direction/.initial=x,
	%
	% the displayed minimum value:
	min/.initial=0,
	% the displayed max value:
	max/.initial=,
	%
	% the width of the entire plot:
	width/.initial=0.6cm,
	%
	% Style options for bullet plots with 'draw direction=x'
	every bullet graph x dir/.style={
		title style={
			at={(0,0)},
			anchor=east,
			align=right,
			%
			text width=4cm,
		},
	},
	% Style options for bullet plots with 'draw direction=y'
	every bullet graph y dir/.style={
		title style={
			at={(0.5,1)},
			anchor=south,
			align=center,
			text width=2cm,
		},
	},
	%
	% Style options for any bullet plot. FIXME : hard to use! Simplify
	% this
	every bullet graph/.style={
		/pgfplots/every axis/.add style={
			/pgfplots/bullet graph/every bullet graph \pgfplots@bullet@dir\space dir,
			scale only axis,
			axis \pgfplots@bullet@other@dir\space line=none,
			enlarge \pgfplots@bullet@other@dir\space limits=false,
			\pgfplots@bullet@dir\space axis line style={opacity=0},
			\pgfplots@bullet@dir tick pos=left,
			\pgfplots@bullet@dir tick align=outside,
			\pgfplots@bullet@other@dir tick=\empty,
			\pgfplots@bullet@other@dir min=\pgfplots@bullet@min@invisible,
			\pgfplots@bullet@other@dir max=\pgfplots@bullet@max@invisible,
			\pgfplots@bullet@other@dir =\pgfkeysvalueof{/pgfplots/bullet graph/width},
		}{},
	},
	bullet graph style/.style={/pgfplots/bullet graph/every bullet graph/.append style={#1}},
	%
	% Style options for the featured measure:
	every performance/.code={%
		% The featured measure should typically be a bar graph.
		% The only difference is when the minimum is > 0, in this case
		% we want to show only a marker.
		\pgfplotscoordmath{default}{parse}{\pgfkeysvalueof{/pgfplots/bullet graph/min}}%
		\pgfplotscoordmath{default}{if is}{\pgfmathresult}{+}{%
			% Ah: min > 0 . then: display a marker.
			\pgfkeysalso{/pgfplots/performance using scatter}%
		}{%
			\pgfkeysalso{/pgfplots/performance using bar}%
		}%
	},
	performance style/.style={/pgfplots/bullet graph/every performance/.append style={#1}},
	%
	% pack these into /pgfplots/ - that simplifies constructions like
	% performance style={performance using projection}
	/pgfplots/performance using projection/.style={
		surf,
		/tikz/line width={0.3333*(\pgfkeysvalueof{/pgfplots/bullet graph/width})},
		%
		shader=flat corner,
		%colormap access=direct,
		point meta=\pgfplots@bullet@dir,%
	},
	/pgfplots/performance using bar/.style={
		/tikz/\pgfkeysvalueof{/pgfplots/bullet graph/draw direction}bar,
		/tikz/bar width={0.3333*(\pgfkeysvalueof{/pgfplots/bullet graph/width})},
		fill=black,
		draw=none,
	},
	/pgfplots/performance using scatter/.style={
		mark=x,
		only marks,
		mark size={0.3333*(\pgfkeysvalueof{/pgfplots/bullet graph/width})},
		mark options={line width=1pt,},
	},
	%
	%
	% Style options for every reference measure:
	every reference/.style={
		% We want centered markers of type '|' .
		%
		% Typically, there is at most one reference. If there are two,
		% the second one should be gray(0.25). To this end, I enabled
		% 'scatter' with 'colormap access=direct':
		/pgfplots/colormap={2 color references}{gray=(0) gray=(0.25)},
		scatter,
		colormap access=direct,
		point meta=\coordindex,
		%
		only marks,
		mark=\if x\pgfplots@bullet@dir|\else -\fi,
		mark size={0.3*(\pgfkeysvalueof{/pgfplots/bullet graph/width})},
		line width=1.5pt,
	},
	reference style/.style={/pgfplots/bullet graph/every reference/.append style={#1}},
	%
	% Style options for the ranges:
	every ranges/.style={
		% Ranges should be shaded areas behind the featured measures.
		%
		% We expect a line plot. If the ranges are 200,250,275, we
		% expect a line plot with <min>,200,250,275,<max>
		%
		% Idea: use a mesh plot with directly mapped colors and
		% colormaps as specified in the "Bullet Graph Design
		% Specification"
		surf,
		line width=\pgfkeysvalueof{/pgfplots/bullet graph/width},
		%
		shader=flat corner,
		colormap access=direct,
		/pgfplots/bullet graph/num coords ranges colormap,
		point meta=\coordindex,
	},
	ranges style/.style={/pgfplots/bullet graph/every ranges/.append style={#1}},
	%
	% These styles are supposed to define the color palettes for
	% 'every ranges'. Note that the color maps are really used as
	% 'palettes', i.e. there will be no interpolation between them.
	2 ranges colormap/.style={/pgfplots/colormap={2 ranges colormap}{gray=(0.65) gray=(0.9)}},
	3 ranges colormap/.style={/pgfplots/colormap={3 ranges colormap}{gray=(0.6) gray=(0.75) gray=(0.9)}},
	4 ranges colormap/.style={/pgfplots/colormap={4 ranges colormap}{gray=(0.5) gray=(0.65) gray=(0.8) gray=(0.9)}},
	5 ranges colormap/.style={/pgfplots/colormap={5 ranges colormap}{gray=(0.5) gray=(0.65) gray=(0.8) gray=(0.9) gray=(0.97)}},
	%
	% auto select one of the available colormaps depending on
	% \numcoords.
	% NOTE: \numcoords is always two bigger than the number of range
	% midpoints!
	num coords ranges colormap/.code={%
		\pgfutil@ifundefined{numcoords}{%
			% Ah - we are in the survey phase. Well, in this case we
			% do not need a colormap anyway. Ignore this.
		}{%
			\c@pgf@countd=\numcoords\relax
			\advance\c@pgf@countd by-1 %
			\def\pgfplots@loc@TMPa{\the\c@pgf@countd}%
			\ifnum\pgfplots@loc@TMPa>5 %
				{%
				\advance\c@pgf@countd by-1 %
				\pgfplotswarning{too many range midpoints}{\the\c@pgf@countd}\pgfeov
				}%
				\def\pgfplots@loc@TMPa{5}%
			\fi
			\pgfplots@log4{Installing \pgfplots@loc@TMPa\space ranges colormap}%
			\pgfplotsset{/pgfplots/bullet graph/\pgfplots@loc@TMPa\space ranges colormap}%
		}%
	},
	%
	% Internal only:
	@set minmax/.style={%
		\pgfkeysvalueof{/pgfplots/bullet graph/draw direction}min=\pgfkeysvalueof{/pgfplots/bullet graph/min},%
		\pgfkeysvalueof{/pgfplots/bullet graph/draw direction}max=\pgfkeysvalueof{/pgfplots/bullet graph/max}%
	},
	/pgfplots/warning/too many range midpoints/.code={%
		\pgfplotsthrow@warning{Found an unexpected number of range midpoints: expected at most 4, but got #1. All ranges after the 4th range midpoint are ignored}%
	},
	/pgfplots/bullet graph/.search also={/pgfplots},
	%
	% unused currently:
	draw position/.initial=0,
}


% Usage:
%
% \tikz
% \bulletgraph[
% 	performance=275,
% 	max=300,
% 	reference=260,
% 	range midpoints={200,250},
% 	title={Revenue 2005 YTD\\[-5pt]\tiny (U.S. \$ in thousands)},
% ];
% 
\def\bulletgraph{\pgfutil@ifnextchar[{\bulletgraph@opt}{\bulletgraph@opt[]}}
\long\def\bulletgraph@opt[#1]#2;{%
	\begingroup
		\pgfqkeys{/pgfplots/bullet graph}{%
			every bullet graph,%
			#1,%
			/pgfplots/bullet graph/@set minmax,%
		}%
		\begin{axis}

		\pgfkeysgetvalue{/pgfplots/bullet graph/performance}\pgfplots@bullet@performance
		\pgfkeysgetvalue{/pgfplots/bullet graph/range midpoints}\pgfplots@bullet@ranges
		\pgfkeysgetvalue{/pgfplots/bullet graph/reference}\pgfplots@bullet@reference
		\pgfkeysgetvalue{/pgfplots/bullet graph/draw direction}\pgfplots@loc@TMPa

		\expandafter\let\expandafter\pgfplots@bulletgraph@scalar\csname pgfplots@bulletgraph@scalar@\pgfplots@loc@TMPa\endcsname
		\expandafter\let\expandafter\pgfplots@bulletgraph@scalarab\csname pgfplots@bulletgraph@scalarab@\pgfplots@loc@TMPa\endcsname

		\ifx\pgfplots@bullet@ranges\pgfutil@empty
		\else
			\pgfplots@bulletgraph@to@coordinates{\pgfkeysvalueof{/pgfplots/bullet graph/min},\pgfplots@bullet@ranges,\pgfkeysvalueof{/pgfplots/bullet graph/max}}%
			\let\pgfplots@bullet@ranges=\pgfplotsretval
			%
			\edef\pgfplots@loc@TMPa{%
				\noexpand\addplot[/pgfplots/bullet graph/every ranges] coordinates {\pgfplots@bullet@ranges};
			}%
			\pgfplots@loc@TMPa
		\fi

		\ifx\pgfplots@bullet@performance\pgfutil@empty
		\else
			\pgfplots@bulletgraph@to@coordinates{\pgfplots@bullet@performance}%
			\let\pgfplots@bullet@performance=\pgfplotsretval
			%
			\edef\pgfplots@loc@TMPa{%
				\noexpand\addplot[/pgfplots/bullet graph/every performance] coordinates {\pgfplots@bullet@performance}
			}%
			\pgfplots@loc@TMPa #2;
		\fi

		\ifx\pgfplots@bullet@reference\pgfutil@empty
		\else
			\pgfplots@bulletgraph@to@coordinates{\pgfplots@bullet@reference}%
			\let\pgfplots@bullet@reference=\pgfplotsretval
			%
			\edef\pgfplots@loc@TMPa{%
				\noexpand\addplot[/pgfplots/bullet graph/every reference] coordinates {\pgfplots@bullet@reference};
			}%
			\pgfplots@loc@TMPa
		\fi
			
		\end{axis}
	\endgroup
}

\def\pgfplots@bulletgraph@scalar@x#1{(#1,\pgfkeysvalueof{/pgfplots/bullet graph/draw position})}
\def\pgfplots@bulletgraph@scalar@y#1{(\pgfkeysvalueof{/pgfplots/bullet graph/draw position},#1)}

\def\pgfplots@bulletgraph@scalarab@x#1#2{(#1,#2)}
\def\pgfplots@bulletgraph@scalarab@y#1#2{(#2,#1)}

% Takes a single element (or a comma-separated list of elements) and
% transforms them into a sequence of (<x>,<y>) ...
\def\pgfplots@bulletgraph@to@coordinates#1{%
	\edef\pgfplotsretval{}%
	\edef\pgfplots@loc@TMPc{#1}%
	\def\pgfplots@loc@TMPa{\pgfplotsforeachungrouped \pgfplots@loc@TMPb in }%
	\expandafter\pgfplots@loc@TMPa\expandafter{\pgfplots@loc@TMPc}{%
		\edef\pgfplotsretval{\pgfplotsretval \pgfplots@bulletgraph@scalar{\pgfplots@loc@TMPb}}%
	}%
}

% Takes a single element (or a comma-separated list of elements) and
% transforms them into a sequence of (<x>,<y>) ...
% UNUSED - is this useful at all!?
\def\pgfplots@bulletgraph@to@rectangles#1{%
	\edef\pgfplotsretval{}%
	\edef\pgfplots@loc@TMPc{\pgfkeysvalueof{/pgfplots/bullet graph/min},#1}%
	\def\pgfplots@loc@TMPa{\pgfplotsforeachungrouped \pgfplots@loc@TMPb in }%
	\let\pgfplots@last=\pgfutil@empty
	\expandafter\pgfplots@loc@TMPa\expandafter{\pgfplots@loc@TMPc}{%
		\ifx\pgfplots@last\pgfutil@empty
		\else
			\edef\pgfplotsretval{\pgfplotsretval 
				\pgfplots@bulletgraph@scalarab{\pgfplots@last}{\pgfplots@bullet@min@invisible}%
				\pgfplots@bulletgraph@scalarab{\pgfplots@loc@TMPb}{\pgfplots@bullet@min@invisible}%
				\pgfplots@bulletgraph@scalarab{\pgfplots@loc@TMPb}{\pgfplots@bullet@max@invisible}%
				\pgfplots@bulletgraph@scalarab{\pgfplots@last}{\pgfplots@bullet@max@invisible}%
			}%
		\fi
		\let\pgfplots@last=\pgfplots@loc@TMPb
	}%
}


\endinput

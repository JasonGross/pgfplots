%--------------------------------------------
%
% Package pgfplots
%
% Provides a user-friendly interface to create function plots (normal
% plots, semi-logplots and double-logplots).
% 
% It is based on Till Tantau's PGF package.
%
% Copyright 2010 by Nick Papior Andersen.
%
% This program is free software: you can redistribute it and/or modify
% it under the terms of the GNU General Public License as published by
% the Free Software Foundation, either version 3 of the License, or
% (at your option) any later version.
% 
% This program is distributed in the hope that it will be useful,
% but WITHOUT ANY WARRANTY; without even the implied warranty of
% MERCHANTABILITY or FITNESS FOR A PARTICULAR PURPOSE.  See the
% GNU General Public License for more details.
% 
% You should have received a copy of the GNU General Public License
% along with this program.  If not, see <http://www.gnu.org/licenses/>.
%
%--------------------------------------------

\newif\ifpgfplots@contour@calc@linear

\usetikzlibrary{calc}

\pgfplots{
    /pgfplots/contour/.is family,
    /pgfplots/calculation method/.is choice,
    /pgfplots/calculation method/linear/.is if=\ifpgfplots@contour@calc@linear,
    /pgfplots/contour/values/.store in=\pgfplots@contour@values,
}


\pgfplots@contour@start{%
    %
    % Determine the right calculation method
    %
    \ifpgfplots@contour@calc@linear
      \let\pgfplots@contour@calc@method=\pgfplots@contour@calc@linear
    \endif
    % create new initial lists % remember!!!!
    \pgfplotsarraysize\pgfplots@contour@levels@array\to\c@pgf@counta
    \c@pgf@countb=0
    \pgfplotsloop{%
        \ifnum\c@pgf@counta=\c@pgfplots@row
            \pgfplotsloopcontinuefalse
        \else
            \pgfplotsloopcontinuetrue
        \fi
    }{%
        \pgfplotsarrayselect\c@pgf@countb\fo\pgfplots@contour@levels@array\to\pgfplots@contour@curlevel
        \pgfplots@contour@calc@method\pgfplots@contour@curlevel%
        \advance\c@pgf@countb by1%
    }
}

\def\pgfplots@contour@calc@linear#1{%
    \c@pgfplots@col=0
    \c@pgfplots@row@end=100
    \c@pgfplots@col@end=100
    \pgfplotsmatrixnewempty\pgfplots@contour@runned@matrix
    \pgfplotsmatrixresize\pgfplots@contour@runned@matrix\c@pgfplots@row@end\c@pgfplots@col@end
    \pgfplotsloop{%
        \ifnum\c@pgfplots@col@end=\c@pgfplots@col
            \pgfplotsloopcontinuefalse
        \else
            \pgfplotsloopcontinuetrue
        \fi
    }{%
        \c@pgfplots@row=0%
        \pgfplotsloop{%
            \ifnum\c@pgfplots@row@end=\c@pgfplots@row
                \pgfplotsloopcontinuefalse
            \else
                \pgfplotsloopcontinuetrue
            \fi
        }{%
            %
            % Begin calculation of the contour, this is not the best implementation but it should work.
            % The wrong doing is looping through the entire matrix for each contour level. This is O(N)*
            %
            % If the matrix element has already been processed then just continue.
            %
            \pgfplots@contour@tmpA
            \expandafter\let\expandafter\pgfplots@contour@tmpA\csname\string\pgfplots@contour@runned@matrix@\c@pgfplots@row,\c@pgfplots@col\endcsname%
            \ifx\pgfplots@contour@tmpA\relax\else
            %
            % First retreive the X and Y points in the quadro point segment of the current row and column.
            % 
            \pgfplots@contour@tmpXA=X[\c@pgfplots@row][\c@pgfplots@col]
            \pgfplots@contour@tmpXB=X[\c@pgfplots@row][\c@pgfplots@col+1]
            \pgfplots@contour@tmpXC=X[\c@pgfplots@row+1][\c@pgfplots@col]
            \pgfplots@contour@tmpXD=X[\c@pgfplots@row+1][\c@pgfplots@col+1]
            \pgfplots@contour@tmpYA=Y[\c@pgfplots@row][\c@pgfplots@col]
            \pgfplots@contour@tmpYB=Y[\c@pgfplots@row][\c@pgfplots@col+1]
            \pgfplots@contour@tmpYC=Y[\c@pgfplots@row+1][\c@pgfplots@col]
            \pgfplots@contour@tmpYD=Y[\c@pgfplots@row+1][\c@pgfplots@col+1]
            
            %
            % Retreive the Z values at the same point and immediately subtract the current level from the point.
            %
            \pgfplots@contour@tmpZA=Z[\c@pgfplots@row][\c@pgfplots@col]-#1
            \pgfplots@contour@tmpZB=Z[\c@pgfplots@row][\c@pgfplots@col+1]-#1
            \pgfplots@contour@tmpZC=Z[\c@pgfplots@row+1][\c@pgfplots@col]-#1
            \pgfplots@contour@tmpZD=Z[\c@pgfplots@row+1][\c@pgfplots@col+1]-#1

            \pgfplots@contour@tmpZA=Z[\c@pgfplots@row][\c@pgfplots@col]-#1
            \pgfplots@contour@tmpZB=Z[\c@pgfplots@row][\c@pgfplots@col+1]-#1
            \pgfplots@contour@tmpZC=Z[\c@pgfplots@row+1][\c@pgfplots@col]-#1
            \pgfplots@contour@tmpZD=Z[\c@pgfplots@row+1][\c@pgfplots@col+1]-#1
            
            
            %\pgfplots@tmpa % tmp counter
            \pgfplots@contour@calc@edge@product\pgplots@contour@tmpZD\pgfplots@contour@tmpZA\to\pgfplots@contour@tmp@edgeA
            \pgfplots@contour@calc@edge@product\pgplots@contour@tmpZA\pgfplots@contour@tmpZB\to\pgfplots@contour@tmp@edgeB
            \pgfplots@contour@calc@edge@product\pgplots@contour@tmpZB\pgfplots@contour@tmpZC\to\pgfplots@contour@tmp@edgeC
            \pgfplots@contour@calc@edge@product\pgplots@contour@tmpZC\pgfplots@contour@tmpZD\to\pgfplots@contour@tmp@edgeD
            \ifnum\pgfplots@contour@tmp@edgeA=\pgfplots@contour@tmp@edgeB\ifnum\pgfplots@contour@edgeC=\pgfplots@contour@edgeD\ifnum\pgfplots@contour@edgeB=\pgfplots@contour@edgeC\ifnum\pgfplots@contour@edgeA=-1.0\else
            % 
            % This means that we have encountered a square where there is a possibility of a contour line 
            %
                
            \fi\fi\fi\fi% the four checks if the edge[A-D] = -1.0
            
            
            
            \fi%
        }%
        \advance\c@pgfplots@row by1
    }%
    \advance\c@pgfplots@col by1
}%


\def\pgfplots@contour@calc@edge@product#1#2\to#3{%
    \pgfmathparse{#1*#2}
    \pgfmathparse{ifthenelse(\pgfmathresult>0,int(-1),ifthenelse(\pgfmathresult==0.0,int(0),int(1)))}\edef#3{\pgfmathresult}
}

\def\pgfplots@contour@cur@point@check#1{%
    \expandafter\let\expandafter#1\csname\string\pgfplots@contour@runned@matrix@\c@pgfplots@row,\c@pgfplots@col\endcsname%
}
\endinput

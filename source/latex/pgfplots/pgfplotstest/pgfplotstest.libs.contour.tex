\testsection{Library: Contour}
\usepgfplotslibrary{contour}
\makeatletter

Testing the math routines.

\pgfplotsmatrixnewempty\pgfplots@data@matrixX
\pgfplotsmatrixnewempty\pgfplots@data@matrixY
\pgfplotsmatrixnewempty\pgfplots@data@matrixZ
\pgfplotsmatrixresize\pgfplots@data@matrixX{20}{20}
\pgfplotsmatrixresize\pgfplots@data@matrixY{20}{20}
\pgfplotsmatrixresize\pgfplots@data@matrixZ{20}{20}
\foreach \x in {0,...,20} {
    \foreach \y in {0,...,20} {
        \pgfmathparse{sin(deg((\x-10)/10*pi))*cos(deg((\y-10)/10*pi))}
        \E\xdef\csname\string\pgfplots@data@matrixX@\x,\y\endcsname{\x}
        \E\xdef\csname\string\pgfplots@data@matrixY@\x,\y\endcsname{\y}
        \E\xdef\csname\string\pgfplots@data@matrixZ@\x,\y\endcsname{\pgfmathresult}
    }
}

\foreach \x in {0,...,20} {
    \foreach \y in {0,...,20} {
        \pgfplotsmatrixvalueofelem{\x},{\y}\of\pgfplots@data@matrixZ,
    }

}

Finding the value of 3,2: \pgfplotsmatrixvalueofelem3,2\of\pgfplots@data@matrixZ


\pgfkeys{/pgfplots/contour/levels={5,10,15}}
The levels of the contour: \pgfplots@contour@levels

\pgfplots@contour@start

\c@pgf@counta=0
\c@pgf@countb=0
\pgfplotsmatrixsize\pgfplots@contour@matrix@points\to\c@pgf@counta\c@pgf@countb
\advance\c@pgf@counta by-1
\advance\c@pgf@countb by-1
\foreach \y in {0,...,\the\c@pgf@counta} {
    \foreach \x in {0,...,\the\c@pgf@countb} {
        \y,\x : \pgfplotsmatrixvalueofelem\y,\x\of\pgfplots@contour@matrix@points
    }

}

\pgfplots@contour@reset@runned

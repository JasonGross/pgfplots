\testsection{Library: Contour}
\usepgfplotslibrary{contour}
\makeatletter

Testing the math routines.

\pgfplotsmatrixnewempty\pgfplots@data@matrixX
\pgfplotsmatrixnewempty\pgfplots@data@matrixY
\pgfplotsmatrixnewempty\pgfplots@data@matrixZ
\pgfplotsmatrixresize\pgfplots@data@matrixX{15}{15}
\pgfplotsmatrixresize\pgfplots@data@matrixY{15}{15}
\pgfplotsmatrixresize\pgfplots@data@matrixZ{15}{15}
\foreach \x in {0,...,15} {
    \foreach \y in {0,...,15} {
        \pgfmathparse{\x/15}
        \E\xdef\csname\string\pgfplots@data@matrixX@\x,\y\endcsname{\pgfmathresult}
        \pgfmathparse{\y/15}
        \E\xdef\csname\string\pgfplots@data@matrixY@\x,\y\endcsname{\pgfmathresult}
        \pgfmathparse{\x*\y/225}
        \E\xdef\csname\string\pgfplots@data@matrixZ@\x,\y\endcsname{\pgfmathresult}
    }
}

\scriptsize
\foreach \x in {0,...,14} {
    \foreach \y in {0,...,14} {
        \pgfplotsmatrixvalueofelem{\x},{\y}\of\pgfplots@data@matrixZ,
    }

}
\normalsize
Finding the value of 3,2: \pgfplotsmatrixvalueofelem3,2\of\pgfplots@data@matrixZ


\pgfkeys{/pgfplots/contour/levels={0.2}}
The levels of the contour: \pgfplots@contour@levels

Test of the interpolation: 
\pgfplots@contour@start

\c@pgf@counta=0
\c@pgf@countb=0
\pgfplotsmatrixsize\pgfplots@contour@matrix@points\to\c@pgf@counta\c@pgf@countb
\advance\c@pgf@counta by-1
\advance\c@pgf@countb by-1

\scriptsize
\foreach \y in {0,...,\the\c@pgf@counta} {
    (\y:0,1) $>$ (\pgfplotsmatrixvalueofelem\y,0\of\pgfplots@contour@matrix@points , \pgfplotsmatrixvalueofelem\y,1\of\pgfplots@contour@matrix@points) = \pgfmathparse{\pgfplotsmatrixvalueofelem\y,0\of\pgfplots@contour@matrix@points *\pgfplotsmatrixvalueofelem\y,1\of\pgfplots@contour@matrix@points}\pgfmathresult

}

\pgfplots@contour@reset@runned
